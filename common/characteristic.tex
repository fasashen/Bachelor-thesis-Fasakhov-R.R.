
{\actuality}
Интерес к изучению движения твердого тела в электрических полях стимулируется многочисленными прикладными задачами в различных областях современной техники. Подобные задачи встают при исследовании бесконтактных опор различных движущихся систем, разработке космических аппаратов~и~др.

Одной из областей, в которой неконтактные подвесы особенно актуальны, является гироскопия, а именно создание неконтактого электростатического подвеса ротора в сферическом гироскопе. Современные гироскопы с неконтактными подвесами - это сложнейшие приборы, которые вобрали в себя новейшие достижения техники. Неконтактные подвесы позволяют существенно увеличить срок службы, уменьшить трение и шум, повысить точность, увеличить рабочие скорости вращения ротора \cite{Electropribor}.

Большой интерес к разработке неконтактных подвесов проявляется как в России, так и за рубежом. Что примечательно, только три страны в мире в настоящее время способны производить электростатические гироскопы. Кроме США и Франции в их число входит и Россия \cite{Electropribor}.

С теоретической точки зрения для решения задачи движения твердого тела в электрических полях необходимо исследование совместной системы уравнений движения твердого тела около неподвижной точки и уравнений электродинамики. Масштабная работа в этом направлении проведена Ю. Г. Мартыненко в его монографии о движении твердых тел в электрических и магнитных полях \cite{Martynenko}. 


{\aim} данной работы является применение конечно элементного метода к решению задачи о сферическом роторе в электростатическом подвесе.
Начиная с простейшей одномерной модели пассивного резонансного электростатического подвеса, провести сравнение различных постановок решения связной задачи электромеханики.

Применяя методы матфизики, асимптотические методы нелинейной механики, динамики твердого тела в ходе работы аналитически оценивается поведение решения. Стоит задача получить численное решение задачи методом конечных элементов (МКЭ), сравнить с аналитическими оценками.


Для~достижения поставленной цели необходимо было решить следующие {\tasks}:
\begin{enumerate}
  \item Исследовать простейшие одноосные некотактные подвесы, провести анализ их устойчивости
  \item Исследовать и сравнить между собой способы моделирования задач электромеханики в системе конечно-элементного анализа ANSYS
  \item Разработать и исследовать, применяя метод конечных элементов,  модель пассивного одноосного подвеса
  \item Применяя методы матфизики, асимптотические методы нелинейной механики, динамики твердого тела аналитически оценить поведение решения конечно-элементной модели
  \item Разработать и исследовать, применяя метод конечных элементов,  модель сферического ротора в трехосном электростатическом подвесе 
\end{enumerate}

{\probation}
Основные результаты работы докладывались~на XIX конференции молодых ученых "Навигация и управление движением" (XIX КМУ 2017) 14-17 марта 2017 г., Санкт-Петербург, Россия.
